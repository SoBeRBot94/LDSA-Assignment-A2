%%%%%%%%%%%%%%%%%%%%%%%%%%%%%%%%%%%%%%%%%
% Uppsala University Assignment Title Page 
% LaTeX Template
% Version 1.0 (27/12/12)
%
% This template has been downloaded from:
% http://www.LaTeXTemplates.com
%
% Original author:
% WikiBooks (http://en.wikibooks.org/wiki/LaTeX/Title_Creation)
% Modified by Elsa Slattegard to fit Uppsala university
% License:
% CC BY-NC-SA 3.0 (http://creativecommons.org/licenses/by-nc-sa/3.0/)

%\title{Title page with logo}
%----------------------------------------------------------------------------------------
%	PACKAGES AND OTHER DOCUMENT CONFIGURATIONS
%----------------------------------------------------------------------------------------

\documentclass[12pt]{article}
\usepackage[english]{babel}
\usepackage[utf8x]{inputenc}
\usepackage{amsmath}
\usepackage{graphicx}
\usepackage{hyperref}
\usepackage{float}
\usepackage[colorinlistoftodos]{todonotes}
\usepackage{listings}
\usepackage{color}
%----------------------------------------------------------------------------------------
%	CODE HILIGHTING CONFIGURATIONS
%----------------------------------------------------------------------------------------
\definecolor{dkgreen}{rgb}{0,0.6,0}
\definecolor{gray}{rgb}{0.5,0.5,0.5}
\definecolor{mauve}{rgb}{0.58,0,0.82}

\lstset{frame=tb,
  language=Bash,
  aboveskip=3mm,
  belowskip=3mm,
  showstringspaces=false,
  columns=flexible,
  basicstyle={\small\ttfamily},
  numbers=none,
  numberstyle=\tiny\color{gray},
  keywordstyle=\color{blue},
  commentstyle=\color{dkgreen},
  stringstyle=\color{mauve},
  breaklines=true,
  breakatwhitespace=true,
  tabsize=3
}

\lstset{frame=tb,
  language=Python,
  aboveskip=3mm,
  belowskip=3mm,
  showstringspaces=false,
  columns=flexible,
  basicstyle={\small\ttfamily},
  numbers=none,
  numberstyle=\tiny\color{gray},
  keywordstyle=\color{blue},
  commentstyle=\color{dkgreen},
  stringstyle=\color{mauve},
  breaklines=true,
  breakatwhitespace=true,
  tabsize=3
}

%----------------------------------------------------------------------------------------


\begin{document}

\begin{titlepage}

\newcommand{\HRule}{\rule{\linewidth}{0.5mm}} % Defines a new command for the horizontal lines, change thickness here

\center % Center everything on the page
 
%----------------------------------------------------------------------------------------
%	HEADING SECTIONS
%----------------------------------------------------------------------------------------

\textsc{\LARGE Uppsala University}\\[1.5cm] % Name of your university/college
\includegraphics[scale=.1]{Uppsala_University_seal_svg.png}\\[1cm] % Include a department/university logo - this will require the graphicx package
\textsc{\Large Large Datasets For Scientific Applications}\\[0.5cm] % Major heading such as course name
\textsc{\large 1TD268}\\[0.5cm] % Minor heading such as course title

%----------------------------------------------------------------------------------------
%	TITLE SECTION
%----------------------------------------------------------------------------------------

\HRule \\[0.4cm]
{ \huge \bfseries Assignment A2}\\[0.4cm] % Title of your document
\HRule \\[1.5cm]
 
%----------------------------------------------------------------------------------------
%	AUTHOR SECTION
%----------------------------------------------------------------------------------------

\begin{minipage}{0.4\textwidth}
\begin{flushleft} \large
\emph{Author:}\\
Sudarsan \textsc{Bhargavan}\\ % Your name
\end{flushleft}

\end{minipage}\\[2cm]

% If you don't want a supervisor, uncomment the two lines below and remove the section above
%\Large \emph{Author:}\\
%John \textsc{Smith}\\[3cm] % Your name

%----------------------------------------------------------------------------------------
%	DATE SECTION
%----------------------------------------------------------------------------------------

{\large \today}\\[2cm] % Date, change the \today to a set date if you want to be precise

\vfill % Fill the rest of the page with whitespace

\end{titlepage}
\section{A - Working with the RDD API}
\subsection{Question A.1}
\begin{enumerate}
\item Read the English transcripts with Spark and count the number of lines.
\newline
\newline
	\textbf{DataSet: Bulgarian dataset(europarl-v7.bg-en.en)}
    \\
    \textbf{Number of Lines:} 406934
\newline
\item  Do the same with the other language (so that you have a separate lineage of RDDs for each).
\newline
\newline
	No of Lines mentioned in the repository document : 406934
    \\
    \\
    \textbf{WordCount} Script
    \begin{lstlisting}
    //WordCount.sh
		#!/bin/bash

		lineCount_en=`/usr/bin/wc --lines ./europarl-v7.bg-en.en`
		lineCount_bg=`/usr/bin/wc --lines ./europarl-v7.bg-en.bg`

		# Print Only The Number of Lines

		output_en=`echo $lineCount_en |/usr/bin/awk '{print $1}'`
		output_bg=`echo $lineCount_bg |/usr/bin/awk '{print $1}'`

		echo "Line Count of English Dataset: $output_en"
		echo
		echo "Line Count of Bulgarian Dataset: $output_bg"
	\end{lstlisting}
   	\textbf{Output}
    \begin{lstlisting}
    	Line Count of English Dataset: 406934

		Line Count of Bulgarian Dataset: 406934
    \end{lstlisting}
\newpage
\item Verify that the line counts are the same for the two languages.
\newline
\newline
	Verifying the line counts for both the languages yields: \cite{DS}
    \newline
    \newline
    \textbf{Bulgarian}
    \newline
		lines\_en = sparkC.textFile("/home/ubuntu/DATA/europarl-v7.bg-en.en") \\
        lines\_en.count() \\
        \textbf{406934}
    \newline
    \newline
    \textbf{English}
    \newline
    	lines\_bg = sparkC.textFile("/home/ubuntu/DATA/europarl-v7.bg-en.bg") \\
        lines\_bg.count() \\
        \textbf{406934}
\newline
\item Count the number of partitions.
\newline
\newline
	Trying to find the number of partitions yields:
    \newline
    \newline
    	lines\_en.getNumPartitioins() \\
        \textbf{2}
\end{enumerate}
\subsection{Question A.2}
\begin{enumerate}
\item Inspect 100 entries from your RDD to verify your pre-processing.
\newline
\newline
	Inspecting 100 entries from RDD
    \newline
    \newline
    \textbf{Output:}
    \newline
    \newline
    \textbf{For The Output Please See The File}
    \newline
    \newline
    \href{run:./inspectedData.txt}{inspectedData.txt}
\newpage
\item Verify that the line counts still match after the pre-processing.
\newline
\newline
	Inspecting line counts after pre-processing still yields the same results for both the languages \\
    \textbf{406934}
\end{enumerate}
\subsection{Question A.3}
\begin{enumerate}
\item Use Spark to compute the 10 most frequently according words in the English language corpus. Repeat for the other language.
\newline
\newline
	\textbf{English}
	\newline
    \newline
    Frequent Words List For English : [('the', 698563), ('of', 362452), ('to', 326291), ('and', 293700), ('in', 222084), ('a', 162764), ('is', 157336), ('that', 155812), ('for', 119429), ('I', 108253)]
	\newline
    \newline
    \textbf{Bulgarian}
    \newline
    \newline
    \textbf{See File Referenced Below}
    \newline
    \href{run:./frequentsBulgarian.txt}{frequentsBulgarian.txt}
\newline
\item Verify that your results are reasonable.
\newline
\newline
	\textbf{After Translation Bulgarian ----- English} \\
    It was found that many matched with the frequent English words. \\
    Please See \href{run:./matchedWords.txt}{Matched Frequent Words}
\end{enumerate}
\newpage
\subsection{Question A.4}
\begin{enumerate}
\item Do your translations seem reasonable? 
\newline
\newline
	While manually comparing with google translate, the translation seemed reasonable. \\
    \newline
    \includegraphics[width=\textwidth]{TaskA_Screenshots/Translations.png}
\end{enumerate}
\hrule
\newpage
\section{B - Working with DataFrames and SQL}
\subsection{Question B.1 - Analysis with DataFrames / SQL}
\begin{enumerate}
\item Which organization has the largest gender pay gap? Which the least?
\newline
\newline
	\textbf{Largest Gender Pay Gap:}
    \newline
    \includegraphics[width=\textwidth]{TaskB_Screenshots/Task_B1_1_a.png}
    \newline
    \newline
    \textbf{Least Gender Pay Gap:}
    \newline
    \includegraphics[width=\textwidth]{TaskB_Screenshots/Task_B1_1_b.png}
\newpage
\item What is the mean gender pay gap across all organization?
\newline
\newline
	\textbf{Mean Gender Pay Gap:}
    \newline
    \includegraphics[width=\textwidth]{TaskB_Screenshots/Task_B1_2.png}
\newline
\newline
\item Export the results of B.1.2 to a CSV file. Inspect the output file to check it looks reasonable.
\newline
\newline
	\textbf{Please See File} \\
    \href{run:./Task_B_1_3.csv}{csv.file}
\newline
\newline
\item What proportion of organizations pay women more than men on average?
\newline
\newline
	\textbf{Proportion of Organization That Pay Women More:}
    \newline
    \includegraphics[width=\textwidth]{TaskB_Screenshots/Task_B1_4.png}
\end{enumerate}
\newpage
\subsection{Question B.2- Advanced DataFrames / SQL}
\begin{enumerate}
\item  Create a new column for the industry sector (for each company) using the SIC code:
\newline
\newline
	The \textbf{broadcast} and \textbf{join} variables were used to modify the \textbf{Data Frame}. Also as per the instructions given the \textbf{sic\_codes} with value \textbf{-1} has been ignored. \\
	\\
	The \textbf{broadcast} variable is used to maintain a read-only cached data of the variable. Data has been joined as per the required conditions, with help of the \textbf{join} command. \cite{BC}
\item Compute the mean gender pay gap per sector.
\newline
\newline
	\textbf{Mean Gender Pay Gap:}
    \newline
    \includegraphics[width=\textwidth]{TaskB_Screenshots/Task_B2_2_a.png}
	\newline
	\newline
	\includegraphics[width=\textwidth]{TaskB_Screenshots/Task_B2_2_b.png}
\item How does gender pay equality compare per sector? Compute some additional statistics.
\newline
\newline
	Calculating the mean values yields the following information : \\
    \newline
    In some cases \textbf{women} were paid more than \textbf{mean}, but in most cases it was the other way around. \\
    \newline
    While calculating median mean per sector the gender pay equality was \textbf{netural}.
    \newline
	\includegraphics[width=\textwidth]{TaskB_Screenshots/Task_B2_3_a.png}
    \newline
    \newline
    While calculating the mean per sector of \textbf{mean bonus pay}, it was found that \textbf{women} were paid more.
    \newline
    \includegraphics[width=\textwidth]{TaskB_Screenshots/Task_B2_3_b.png}
    \newline
    \newline
    While calculating the median per sector of \textbf{mean bonus pay}, a lot of negative values were found, which according to the references provided, means that \textbf{women} were paid more. \cite{GPGS}
    \newline
    \includegraphics[width=\textwidth]{TaskB_Screenshots/Task_B2_3_c.png}
\end{enumerate}
\hrule
\newpage
\section{C - Spark Clusters and Deployment}
\begin{enumerate}
\item Modify a copy of your code from Section A, so that it runs on your cluster.
\newline
\newline
	\textbf{Run Jobs - Cluster Mode}
    \newline
    	In order to run a pyspark job in the cluster, the spark master url has to passed to the \textbf{SparkContext} method. \\
    \begin{lstlisting}
    //PySparkJobClusterMode.py
        	#!/usr/bin/env python3
        	import pyspark as pys
            sparkC = pys.SparkContext("spark://localhost:7077")
    \end{lstlisting}
\item Run your code first without and then with .cache() - and look under the storage tab in the web GUI for your application. What do you notice? Explain briefly what’s going on.
\newline
\newline
	\textbf{Cache}
    	When the cache method is used, the \textbf{RDD} caches a copy of the imported data, for further operations. \\
        When we omit the cache method, then the \textbf{RDD} waits for an event to get triggered, after which it loads the data. \cite{CA}
	\newline
    \newline
	\includegraphics[width=\textwidth]{TaskC_Screenshots/Task_C_4.png}
\newpage
\item Use the Web GUI to explore your cluster and examine jobs, stages, and tasks. Create an example that requires a job with more than one stage. Explain, with reference to the Spark API methods you invoke in your code, why this is so.
\newline
\newline
	\textbf{Multi-Stage Jobs}
    \newline
    	A stage is a smaller set of tasks from a job. Stages can be parallelized if they are independent transformations are actions. \cite{ST}
	\newline
    \newline
    	Here the \textbf{Task A3} has multiple-stages. But they cannot be parallelized because each stage is dependent on each other. \textbf{Stage Id: 0} represents the reduce operation. \textbf{Stage Id: 1} represents the sort operation.
     \newline
     \newline
     \includegraphics[width=\textwidth]{TaskC_Screenshots/Task_C_5.png}
\end{enumerate}
\hrule
\newpage
%-------------------------
\bibliography{sources}
\bibliographystyle{IEEEtran}
\end{document}
